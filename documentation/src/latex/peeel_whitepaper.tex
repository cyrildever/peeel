% PEEEL white paper

%----------------------------------------------------------------------------------------
%	PACKAGES AND OTHER DOCUMENT CONFIGURATIONS
%----------------------------------------------------------------------------------------

\documentclass[twoside,twocolumn]{article}

\usepackage{blindtext} % Package to generate dummy text throughout this template 

\usepackage[sc]{mathpazo} % Use the Palatino font
\usepackage[T1]{fontenc} % Use 8-bit encoding that has 256 glyphs
%\linespread{1.05} % Line spacing - Palatino needs more space between lines
\usepackage{microtype} % Slightly tweak font spacing for aesthetics
\usepackage{eufrak}
\usepackage{graphicx} % For \scalebox

\usepackage[english]{babel} % Language hyphenation and typographical rules

\usepackage[hmarginratio=1:1,top=32mm,columnsep=20pt]{geometry} % Document margins
\usepackage[hang, small,labelfont=bf,up,textfont=it,up]{caption} % Custom captions under/above floats in tables or figures
\usepackage{booktabs} % Horizontal rules in tables

\usepackage{lettrine} % The lettrine is the first enlarged letter at the beginning of the text

\usepackage{enumitem} % Customized lists
\setlist[itemize]{noitemsep} % Make itemize lists more compact

\usepackage{abstract} % Allows abstract customization
\renewcommand{\abstractnamefont}{\normalfont\bfseries} % Set the "Abstract" text to bold
\renewcommand{\abstracttextfont}{\normalfont\small\itshape} % Set the abstract itself to small italic text

\usepackage{titlesec} % Allows customization of titles
\renewcommand\thesection{\Roman{section}} % Roman numerals for the sections
\renewcommand\thesubsection{\arabic{subsection}} % roman numerals only for subsections
\titleformat{\section}[block]{\Large\scshape\centering}{\thesection.}{1em}{} % Change the look of the section titles
\titleformat{\subsection}[block]{\large\scshape}{\thesubsection.}{1em}{} % Change the look of the section titles

\usepackage{fancyhdr} % Headers and footers
\pagestyle{fancy} % All pages have headers and footers
\fancyhead{} % Blank out the default header
\fancyfoot{} % Blank out the default footer
\fancyhead[C]{Compact Identity Footprint $\bullet$ Cyril Dever} % Custom header text
\fancyfoot[RO,LE]{\thepage} % Custom footer text
\setlength{\headheight}{14pt}

\usepackage{titling} % Customizing the title section

\usepackage{hyperref} % For hyperlinks in the PDF

\usepackage[symbol]{footmisc} % To use special character in footnote
\renewcommand{\thefootnote}{\arabic{footnote}}

\usepackage{outlines}
\usepackage[font=itshape]{quoting}

\usepackage[linesnumbered,ruled,vlined]{algorithm2e}
\SetKw{Continue}{continue}
\SetKw{KwBy}{by}

%----------------------------------------------------------------------------------------
%	FUNCTIONS
%----------------------------------------------------------------------------------------

\newcommand{\ceil}[1]{\left\lceil #1 \right\rceil}
\newcommand{\floor}[1]{\left\lfloor #1 \right\rfloor}
\newcommand{\bsfnote}{\textsuperscript{*}} % for reference to the base64 string note
\newcommand{\hexnote}{\textsuperscript{$\dagger$}} % for reference to the hex string note
\newcommand{\mod}[1]{\ \mathrm{mod}\ #1}

%----------------------------------------------------------------------------------------
%	LISTINGS
%----------------------------------------------------------------------------------------

\usepackage{amsthm}
\theoremstyle{definition}
\newtheorem{definition}{Definition}

\theoremstyle{remark}
\newtheorem*{remark}{Note}
\newtheorem*{recall}{Recall}

%----------------------------------------------------------------------------------------
%	FIGURES
%----------------------------------------------------------------------------------------

\usepackage{tikz}
\usepackage{caption}

\usetikzlibrary{shapes.geometric, arrows, calc, positioning}

\tikzstyle{startstop} = [rectangle, rounded corners, minimum width=3cm, minimum height=1cm,text centered, draw=black]
\tikzstyle{io} = [trapezium, trapezium left angle=70, trapezium right angle=110, minimum width=3cm, minimum height=1cm, text centered, text width=1.7cm, inner sep=0.4cm, draw=black]
\tikzstyle{process} = [rectangle, minimum width=3cm, minimum height=1cm, text centered, draw=black]
\tikzstyle{decision} = [diamond, minimum width=3cm, minimum height=1cm, text centered, inner sep=-0.1cm, draw=black]
\tikzstyle{arrow} = [thick,->,>=stealth]
\tikzset{XOR/.style={draw,circle,append after command={
        [shorten >=\pgflinewidth, shorten <=\pgflinewidth,]
        (\tikzlastnode.north) edge (\tikzlastnode.south)
        (\tikzlastnode.east) edge (\tikzlastnode.west)
        }
    }
}

%----------------------------------------------------------------------------------------
%	TITLE SECTION
%----------------------------------------------------------------------------------------

\usepackage[english]{datetime2}
\DTMsavedate{thedate}{2020-09-23}

\setlength{\droptitle}{-5\baselineskip} % Move the title up

\pretitle{\begin{center}\Large\bfseries}
\posttitle{\end{center}}
\title{Compact Identity Footprint} % Title
\author{%
    \textsc{Cyril Dever}\\ % Name
    \normalsize Edgewhere \\ % Institution
}
% \date{\today} % Leave empty to omit a date
\date{\DTMusedate{thedate}}
\renewcommand{\maketitlehookd}{%
    \begin{abstract}
        \noindent We define a compact identity footprint for anyone who needs to have compact pseudonymized representation of an identity. 
        It should be used in conjunction with any other type of data footprint to associate the latter with its rightful owner.
    \end{abstract}
}

%----------------------------------------------------------------------------------------

\begin{document}

% Print the title
\maketitle

%----------------------------------------------------------------------------------------
%	ARTICLE CONTENTS
%----------------------------------------------------------------------------------------

\section{Introduction}

\lettrine[nindent=0em,lines=3]{S}ometimes, one needs a way to compare data without having access to its source. Usually, the use of any cryptographic 
hash function would suffice. But what if you also need to be sure of its ownership?

The present document describes a simple yet effective way to build such a simple "owned" footprint.
We call it the Peeel\texttrademark.

%----------------------------------------------------------------------------------------

\section{Quick Description}

The Peeel\texttrademark~algorithm is used to create a unique string from contact data that may eventually serve as some proof of ownership to another 
data with which it would be associated.

It follows the simple rules below:
\begin{itemize}
    \item Each source data must first be normalized according to the \emph{Empreinte Sociométrique} standards \cite{empreinteSociometrique:cyd};
    \item The normalized data is then hashed using the \texttt{SHA-256} algorithm applied twice;
    \item The hashed data is then concatenated in the exact order of the complementaries code, eg. hashed date of birth then hashed firstname 
        for a \texttt{"dob+firstname"} code;
    \item This concatenation is next itself hashed using the \texttt{SHA-256} algorithm twice to form the final hash data.
\end{itemize}

The final result is a tuple of the hexadecimal string representation of the returned hash and the set of codes used,
eg. $$(\texttt{29023e[\dots]e9a2}, \{ \texttt{firstname}, \texttt{dob} \})$$

\begin{definition}[Complementary Data]
    \label{complementaryData}
    We call complementary data $\chi_i$ of an identity $i$ each different contact data that would be used to build his Peeel.
    It is defined by the tuple of its code $\chi^c$ and its associated data source $\chi^d$, eg.$$
        \chi_i \gets \left\{
            \begin{array}{l}
                \chi_i^c := \texttt{firstname} \\ \\
                \chi_i^d := \textrm{Cyril} \\
            \end{array}
        \right.
    $$

    As of this version, the available complementary data codes could be one of the following:
    \begin{itemize}
        \item \texttt{dob}: a date of birth (respecting the ISO format, ie. \texttt{YYYYMMDD});
        \item \texttt{gender}: $1$ for male, $2$ for female, or $0$ for unknown or indeterminate;
        \item \texttt{firstname}: a first name;
        \item \texttt{lastname}: a last name;
        \item \texttt{middle}: the middle names or initials (eg. the \emph{F.} in \emph{John F. Kennedy}).
    \end{itemize}

    We define this set of available codes as $\mathcal{\textbf{C}}$.
\end{definition}

%----------------------------------------------------------------------------------------

\section{Algorithm}

Let $N_{ES}: \omega \to \omega$ be the normalizing function through the Empreinte Sociométrique standard for the input data\footnote{for example, our 
Javascript \texttt{es-normalizer} library: \small\url{https://npmjs.com/package/es-normalizer}}, and $\mathfrak{h}()$ the cryptographic hashing 
function\footnote{we use \texttt{SHA-256} for its wide adoption in all the major browsers}.

And let $\chi_1, \chi_2, \dots, \chi_n$ be an ordered set of complementary data.

Algorithm \ref{algo:peeel} describes how a Peeel is built.
\begin{algorithm}
    \SetKwProg{throw}{throw}{}{}
    \KwIn{A set of complementary data $\{ \chi_1, \chi_2, \dots, \chi_n \}$}
    \KwOut{The hashed data and its set of used codes, or an error}
    initialize the set of normalized hashed items $\mathcal{H} \gets \emptyset$,
        and the ordered set of kept codes $\mathcal{C}_{k} \gets \emptyset$; \\
    \For{$i \gets 0$ \KwTo $n$ \KwBy $1$}{
        \If{$\chi_i^c \not\in \mathcal{\textbf{C}}$}{
            \Continue;
        }
        $\mathcal{H} \gets \mathfrak{h}\left( N_{ES}(\chi_i^d) \right)$; \\
        $\mathcal{C}_{k} \gets \chi_i^c$; \\
    }
    \If{$|\mathcal{H}| = 0 \vee |\mathcal{K}| \neq |\mathcal{H}|$}{
        \throw{invalid input}{}
    }
    initialize the concatenation string $S$; \\
    \For{$i \gets 0$ \KwTo $|\mathcal{H}|$ \KwBy $1$}{
        $S \gets S \mathbin\Vert \mathcal{H}_i$; \\
    }
    build the hexadecimal string $hexStr := \Big( \mathfrak{h}(S) \Big)_{16}$ \\
    \Return{$(hexStr, \mathcal{C}_{k})$;}
    \caption{Peeel\texttrademark~algorithm}
    \label{algo:peeel}
\end{algorithm}

The returned set of used codes should be in the same order than the input set of complementary data codes.
In other words, if the input set is totally valid, we should have:$$
    \mathcal{C}_{k} := \{ \chi_1^c, \chi_2^c, \dots, \chi_n^c \}
$$

%----------------------------------------------------------------------------------------

\section{Usage}

The main usage of the Peeel\texttrademark~algorithm is in association with another data to make it a kind of proof of ownership of this other data 
in full respect of data privacy regulations such as GDPR or CCPA.

For example, in our Consent Management Blockchain\cite{drd2:cyd}, we use it to link the consented data with the recorded transaction.
That way, whenever a data is consented, we can make sure that the person who consents is the current rightful owner of the data and that further use 
of the data by an eventual previous owner can't be possible anymore. And if we want proof of ownership, we just need to implement some sort of 
questionnaire to rebuild the Peeel (without even knowing anything about it except for the codes and therefore the kind of question to ask)\footnote{
    You might want to check out our Fruuut\texttrademark~library: \url{https://github.com/cyrildever/fruuut}, to see our last implementation of 
    such a questionnaire.
}.

% \vfill\eject % To force break column if need be
% \tableofcontents % Uncomment to add a table of contents

%----------------------------------------------------------------------------------------
%	REFERENCE LIST
%----------------------------------------------------------------------------------------

\begin{thebibliography}{99} % Bibliography

\bibitem[1]{empreinteSociometrique:cyd}
Cyril Dever. \emph{Système de traitement d'une base de données personnelle par un opérateur extérieur}, patent pending \texttt{FR1905778}, 2019.

\bibitem[2]{drd2:cyd}
Cyril Dever. \emph{Deferred Resolution Consensus Protocol on Double-Helix Double-Circuit Blockchains}, 2020.

\end{thebibliography}

%----------------------------------------------------------------------------------------

\end{document}